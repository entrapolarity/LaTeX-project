\documentclass{article}
\usepackage[utf8]{inputenc}
\usepackage{graphicx}
\usepackage[colorlinks,allcolors=red]{hyperref}
\graphicspath{ {./images/} }

\title{Eurovision Survey}
\author{Anastasiia Kornilova}
\date{June 2022}

\begin{document}

\maketitle

\section{Introduction}

The Eurovision Song Contest 2022 was the 66th edition of the Eurovision Song Contest. It took place in Turin, Italy, following the country's victory at the 2021 contest with the song "Zitti e buoni" by Måneskin. Organised by the European Broadcasting Union (EBU) and host broadcaster Radiotelevisione italiana (RAI), the contest was held at the PalaOlimpico, and consisted of two semi-finals on 10 and 12 May, and a final on 14 May 2022. The three live shows were presented by Italian television presenter Alessandro Cattelan, Italian singer Laura Pausini and Lebanese-British singer Mika. See Figure \ref{figure_1}.

Forty countries participated in the contest, with Armenia and Montenegro returning after their absences from the previous edition. Russia had originally planned to participate, but was excluded due to its invasion of Ukraine.

The winner was Ukraine with the song "Stefania", performed by Kalush Orchestra and written by the group's members Ihor Didenchuk, Ivan Klimenko, Oleh Psiuk, Tymofii Muzychuk and Vitalii Duzhyk. Ukraine's 439 points received from the televote in the final are the most televoting points received in the contest's history to date, making "Stefania" the first song sung entirely in Ukrainian and the first song with hip-hop elements to win the contest. The United Kingdom, Spain, Sweden, and Serbia rounded out the top five, with the United Kingdom and Spain achieving their best results since 1998 and 1995 respectively, and Serbia achieving its best result since 2012. It was also a record-extending sixteenth time that the United Kingdom finished in second place. Italy finished in sixth place, thereby achieving the best result for a host country since 2016. See Figure \ref{figure_2}.

The EBU reported that the contest had a television audience of 161 million viewers in 34 European markets, a decrease of 22 million viewers from the previous edition, however, it is noted that this is due to the exclusion of Russia and the lack of audience figures from Ukraine, with the overall figures up by 7 million viewers in a comparable market from 2021. An increase of three percent in the 15–24 year old age range was also reported. A total of 18 million viewers watched the contest online on YouTube and TikTok. (\hyperlink{Ginsburgh 2008}{Ginsburgh 2008}).

\begin{figure}
\caption{The correlation between order and the rank of the artist.}
\includegraphics[width=0.90\linewidth]{images/scatterplot.jpg}
\label{figure_1}
\end{figure}

\begin{figure}
\caption{The visualization of the ranks.}
\includegraphics[width=0.90\linewidth]{images/heatmap.jpg}
\label{figure_2}
\end{figure}

\section{Bibliography}

\begin{thebibliography}{9}
\bibitem{Ginsburgh 2008} \hypertarget{Ginsburgh 2008}{}
Ginsburgh V., Noury A. G. The Eurovision song contest. Is voting political or cultural? // European Journal of Political Economy. 2008.

\end{thebibliography}

\end{document}
